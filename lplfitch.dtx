% \iffalse meta-comment
%
% Copyright (C) 2013 by Dave Barker-Plummer, John Etchemendy,
% and Richard Zach
% -------------------------------------------------------
% 
% This file may be distributed and/or modified under the
% conditions of the LaTeX Project Public License, either version 1.2
% of this license or (at your option) any later version.
% The latest version of this license is in:
%
%    http://www.latex-project.org/lppl.txt
%
% and version 1.2 or later is part of all distributions of LaTeX 
% version 1999/12/01 or later.
%
% \fi
%
% \iffalse
%<*driver>
\ProvidesFile{lplfitch.dtx}
%</driver>
%<package>\NeedsTeXFormat{LaTeX2e}[1999/12/01]
%<package>\ProvidesPackage{lplfitch}
%<*package>
    [2013/05/07 v1.0 LPL Fitch style]
%</package>
%
%<*driver>
\documentclass{ltxdoc}
\usepackage{lplfitch}[2013/05/07]
\EnableCrossrefs         
\CodelineIndex
\RecordChanges
\begin{document}
  \DocInput{lplfitch.dtx}
  \PrintChanges
  \PrintIndex
\end{document}
%</driver>
% \fi
%
% \CheckSum{0}
%
% \CharacterTable
%  {Upper-case    \A\B\C\D\E\F\G\H\I\J\K\L\M\N\O\P\Q\R\S\T\U\V\W\X\Y\Z
%   Lower-case    \a\b\c\d\e\f\g\h\i\j\k\l\m\n\o\p\q\r\s\t\u\v\w\x\y\z
%   Digits        \0\1\2\3\4\5\6\7\8\9
%   Exclamation   \!     Double quote  \"     Hash (number) \#
%   Dollar        \$     Percent       \%     Ampersand     \&
%   Acute accent  \'     Left paren    \(     Right paren   \)
%   Asterisk      \*     Plus          \+     Comma         \,
%   Minus         \-     Point         \.     Solidus       \/
%   Colon         \:     Semicolon     \;     Less than     \<
%   Equals        \=     Greater than  \>     Question mark \?
%   Commercial at \@     Left bracket  \[     Backslash     \\
%   Right bracket \]     Circumflex    \^     Underscore    \_
%   Grave accent  \`     Left brace    \{     Vertical bar  \|
%   Right brace   \}     Tilde         \~}
%
%
% \changes{v0.1}{Aeons ago}{Initial version}
% \changes{v1.0}{2012/05/07}{First public, documented version}
%
% \GetFileInfo{lplfitch.dtx}
%
% \DoNotIndex{\newcommand,\newenvironment}
% 
%
% \title{The \textsf{lplfitch} package\thanks{This document
%   corresponds to \textsf{lplproof}~\fileversion, dated \filedate.}}
% \author{John Etchemendy \and Dave Barker-Plummer \and Richard Zach}
%
% \maketitle
%
% \section{Introduction}
%
% The package |lplfitch| provides macros for typesetting natural
% deduction proofs in ``Fitch'' style, with subproofs indented and
% offset by scope lines.  It produces proofs in the format used in the
% textbook \textit{Languege, Proof, and Logic} by Jon Barwise and John
% Etchemendy.  The package was originally written by John Etchemendy, and 
% incorporates changes by Dave Barker-Plummer and Richard Zach.
%
%\section{Fitch Proofs by Example}
%\label{sect:proofs-by-example}
%
%In this section we will describe how to typeset the Fitch-style
%natural deduction proof shown in figure~\ref{fig:example-proof}.  This
%proof uses all of the macros that we have developed for typesetting
%proofs and therefore serves as a good illustration.
%
%\begin{figure}
%\fitchprf
%  {}
%  {
%   \subproof{\nline[1.]{\uni{x}{(Cube(x)\lif Small(x))}}}
%            {
%             \subproof{\nline[2.]{\exi{x}{Cube(x)}}}
%                      {
%                       \boxedsubproof[3.]{a}{Cube(a)}
%                        {
%                         \jline[\lalle{1}]{[4.]{Cube(a)\lif Small(a)}}\\
%                         \jline[\life{4}{3}]{[5.]{Small(a)}}\\
%                         \jline[\lexii{5}]{[6.]{\exi{x}{Small(x)}}}
%                        }
%                        \jline[\lexie{2}{3--6}]{[7.]{\exi{x}{Small(x)}}}
%                      }
%                    \jline[\lifi{2--7}]{[8.]{\exi{x}{Cube(x)}\lif
%					\exi{x}{Small(x)}}}
%                   }
%             \jline[\lifi{1--8}]{[9.]{\brokenform{(\uni{x}{(Cube(x)\lif
%Small(x))}\lif}{\formula{(\exi{x}{Cube(x)}\lif\exi{x}{Small(x)})}}}} 
%  }
%\caption{An Example Proof}
%\label{fig:example-proof}
%\end{figure}
%
%
%The commands for producing this proof are given
%in figure~\ref{fig:proof-eg-cmds}.  This proof is a little hard to take
%in all at once, and so in the remainder of this section we will describe
%the development of this proof, which we hope is of use to readers.
%
%\begin{figure}
%\begin{verbatim}
%\fitchprf
%  {}
%  {
%   \subproof{\nline[1.]{\uni{x}{(Cube(x)\lif Small(x))}}}
%            {
%             \subproof{\nline[2.]{\exi{x}{Cube(x)}}}
%                      {
%                       \boxedsubproof[3.]{a}{Cube(a)}
%                        {
%                         \jline[\lalle{1}]{[4.]{Cube(a)\lif Small(a)}}\\
%                         \jline[\life{4}{3}]{[5.]{Small(a)}}\\
%                         \jline[\lexii{5}]{[6.]{\exi{x}{Small(x)}}}
%                        }
%                        \jline[\lexie{2}{3--6}]{[7.]{\exi{x}{Small(x)}}}
%                      }
%                    \jline[\lifi{2--7}]{[8.]{\exi{x}{Cube(x)}\lif
%                                                 \exi{x}{Small(x)}}}
%                   }
%             \jline[\lifi{1--8}]{[9.]{\brokenform{(\uni{x}{(Cube(x)\lif
%Small(x))}\lif}{\formula{(\exi{x}{Cube(x)}\lif\exi{x}{Small(x)})}}}} 
%  }
%\end{verbatim}
%\caption{The commands for producing the proof of
%figure~\ref{fig:example-proof}}
%\label{fig:proof-eg-cmds}
%\end{figure}
%
%We recommend developing the commands to typeset a proof from the inside
%out, starting with the most deeply nested subproof.  We also recommend
%ignoring justifications until the proof is complete, primarily for
%simplicity but also because the line numbers are not known until the
%complete proof structure is present.
%
%We begin by writing the lines of the innermost subproof (lines 3-6 of the
%main proof).
%
%\begin{quote}
%\begin{tabular}{ll}
%\verb+\nline{Cube(a)}+ & \formula{Cube(a)}\\
%\verb+\nline{Cube(a)\lif Small(a)}+ & \formula{Cube(a)\lif Small(a)}\\
%\verb+\nline{Small(a)}+ & \formula{Small(a)}\\
%\verb+\nline{\exi{x}{Small(x)}}+ & \formula{\exi{x}{Small(x)}}
%\end{tabular}
%\end{quote}
%
%The macro \verb+\nline+ produces a single line of the proof.  The mandatory 
%argument to \verb+\nline+ is the formula to appear within the step.  You will notice
%that we have used the macro \verb+\lif+ to produce an implication arrow,
%and \verb+\exi+ to produce an existentially quantified formula.  These
%are members of a suite of macros that we find convenient for producing
%formulae.  These are described in section~\ref{sect:formulae}. 
%
%The \verb+\nline+ macro may only be used inside a proof context.  We construct such a
%context by using the \verb+\fitchprf+ macro.  This macro has two mandatory
%arguments, corresponding to the premises and the body of the proof.  Each argument
%is a list of lines, separated by the \verb+\\+ command.  We therefore embed the
%lines of our example proof like this:
%
%\begin{verbatim}
%\fitchprf{\nline{Cube(a)}}
%         {
%          \nline{Cube(a)\lif Small(a)}\\
%          \nline{Small(a)}\\
%          \nline{\exi{x}{Small(x)}}
%         }
%\end{verbatim}
%
%\noindent typesetting a document containing this command results in the proof:
%\begin{quote}
%\fitchprf{\nline{Cube(a)}}
%         {
%          \nline{Cube(a)\lif Small(a)}\\
%          \nline{Small(a)}\\
%          \nline{\exi{x}{Small(x)}}
%         }
%\end{quote}
%
%Our next step is to embed this proof as a subproof within the larger proof.  This is
%simply achieved.  We first change the use of \verb+\fitchprf+ to a
%\verb+\subproof+ command, and then treat this as a line of the body of another
%\verb+\fitchprf+ command.  The result of doing this is:
%
%\begin{verbatim}
%\fitchprf{\nline{\exi{x}{Cube(x)}}}
%         {
%          \subproof{\nline{Cube(a)}}
%                   {
%                    \nline{Cube(a)\lif Small(a)}\\
%                    \nline{Small(a)}\\
%                    \nline{\exi{x}{Small(x)}}
%                   }
%          \nline{\exi{x}{Small(x)}}
%         }
%\end{verbatim}
%
%\noindent which yields the proof:
%
%\begin{quote}
%\fitchprf{\nline{\exi{x}{Cube(x)}}}
%         {
%          \subproof{\nline{Cube(a)}}
%                   {
%                    \nline{Cube(a)\lif Small(a)}\\
%                    \nline{Small(a)}\\
%                    \nline{\exi{x}{Small(x)}}
%                   }
%          \nline{\exi{x}{Small(x)}}
%         }
%\end{quote}
%
%\noindent The new outer \verb+\fitchprf+ has two elements in the second
%argument, one of which is the subproof that we just wrote and the second is a
%formula.  As we embedded the subproof in the body of the main proof we
%changed the command from \verb+\fitchprf+ to \verb+\subproof+, the definitions of
%these two macros are almost identical but for the adjustment of vertical spacing
%after the use of a \verb+\subproof+ command.  Note that no \verb+\\+ command is
%required after the use of a \verb+\subproof+ command.
%
%Two further applications of this technique give us the command:
%
%\begin{verbatim}
%\fitchprf{}
%         {
%          \subproof{\nline{\uni{x}{(Cube(x)\lif Small(x))}}}
%                   {
%                    \subproof{\nline{\exi{x}{Cube(x)}}}
%                             {
%                              \subproof{\nline{Cube(a)}}
%                                       {
%                                        \nline{Cube(a)\lif Small(a)}\\
%                                        \nline{Small(a)}\\
%                                        \nline{\exi{x}{Small(x)}}
%                                       }
%                              \nline{\exi{x}{Small(x)}}
%                             }
%                    \nline{\exi{x}{Cube(x)\lif\exi{x}{Small(x)}}}
%                   }
%                   \nline{\brokenform{(\uni{x}{(Cube(x)\lif Small(x))}\lif}
%                              {\formula{(\exi{x}{Cube(x)}\lif
%                                                  \exi{x}{Small(x)})}}}
%         }
%\end{verbatim}
%
%\noindent which produces the proof
%
%\begin{quote}
%\fitchprf{}
%         {
%          \subproof{\nline{\uni{x}{(Cube(x)\lif Small(x))}}}
%                   {
%                    \subproof{\nline{\exi{x}{Cube(x)}}}
%                             {
%                              \subproof{\nline{Cube(a)}}
%                                       {
%                                        \nline{Cube(a)\lif Small(a)}\\
%                                        \nline{Small(a)}\\
%                                        \nline{\exi{x}{Small(x)}}
%                                       }
%                              \nline{\exi{x}{Small(x)}}
%                             }
%                    \nline{\exi{x}{Cube(x)\lif\exi{x}{Small(x)}}}
%                   }
%          \nline{\brokenform{(\uni{x}{(Cube(x)\lif
%Small(x))}\lif}{\formula{(\exi{x}{Cube(x)}\lif\exi{x}{Small(x)})}}}
%         }
%\end{quote}
%
%Notice that the last line contains a long formula, and I have used the
%\verb+\brokenform+ command to typeset this formula.  The \verb+\brokenform+ command
%takes two arguments, the first is the first part of the formula, which will be
%typeset left justified in the space available, while the second argument contains
%the remaining pieces of the formula which will appear in subsequent lines.  These
%lines should be separated by uses of the \verb+\\+ command, and each line should
%(usually) be a \verb+\formula+.  These lines will be typeset right justified in the
%available space.
%
%\subsection{Adding line numbers}
%
%The \verb+\nline+ command takes an optional argument which is the number of the line
%within the proof (\verb+\nline+ stands for {\em numbered line}).  We can simply add
%these now that we have all of the lines of the proof.
%
%\begin{verbatim}
%\fitchprf{}
%         {
%          \subproof{[1.]\nline{\uni{x}{(Cube(x)\lif Small(x))}}}
%                   {
%                    \subproof{\nline[2.]{\exi{x}{Cube(x)}}}
%                             {
%                              \subproof{\nline[3.]{Cube(a)}}
%                                       {
%                                        \nline[4.]{Cube(a)\lif Small(a)}\\
%                                        \nline[5.]{Small(a)}\\
%                                        \nline[6.]{\exi{x}{Small(x)}}
%                                       }
%                              \nline[7.]{\exi{x}{Small(x)}}
%                             }
%                    \nline[8.]{\exi{x}{Cube(x)\lif\exi{x}{Small(x)}}}
%                   }
%          \nline[9.]{\brokenform{(\uni{x}{(Cube(x)\lif
%Small(x))}\lif}{\formula{(\exi{x}{Cube(x)}\lif\exi{x}{Small(x)})}}}
%         }
%\end{verbatim}
%\noindent which results in the proof:
%\begin{quote}
%\fitchprf{}
%         {
%          \subproof{\nline[1.]{\uni{x}{(Cube(x)\lif Small(x))}}}
%                   {
%                    \subproof{\nline[2.]{\exi{x}{Cube(x)}}}
%                             {
%                              \subproof{\nline[3.]{Cube(a)}}
%                                       {
%                                        \nline[4.]{Cube(a)\lif Small(a)}\\
%                                        \nline[5.]{Small(a)}\\
%                                        \nline[6.]{\exi{x}{Small(x)}}
%                                       }
%                              \nline[7.]{\exi{x}{Small(x)}}
%                             }
%                    \nline[8.]{\exi{x}{Cube(x)\lif\exi{x}{Small(x)}}}
%                   }
%          \nline[9.]{\brokenform{(\uni{x}{(Cube(x)\lif
%Small(x))}\lif}{\formula{(\exi{x}{Cube(x)}\lif\exi{x}{Small(x)})}}}
%         }
%\end{quote}
%
%In the eventual proof, a boxed constant is introduced at line 3.  We can convert
%the simple subproof into a boxed subproof.  We use the \verb+\boxedsubproof+
%macro to achieve this.  The conversion is straightforward, we replace the
%first argument of the \verb+\subproof+ command which was
%\verb+{\nline[3.]{Cube(a)}}}+ by the three arguments \verb+[3.]{a}{Cube(a)}+
%obtaining
%
%\begin{verbatim}
%\boxedsubproof[3.]{a}{Cube(a)}
%         {
%          \nline[4.]{Cube(a)\lif Small(a)}\\
%          \nline[5.]{Small(a)}\\
%          \nline[6.]{\exi{x}{Small(x)}}
%         }
%\end{verbatim}
%
%\subsection{Adding Justifications}
%
%The final thing that we need to do to complete the desired proof is the addition of
%justifications.  A justified line is produced using the \verb+\jline+ macro, which
%has one mandatory and one optional argument.  The optional argument contains the
%justification, while the mandatory argument must contain {\em all} of the arguments
%previously given to \verb+\nline+.  So the command 
%\begin{quote}
%\verb+\nline[5.]{Small(a)}+
%\end{quote} 
%\noindent produces exactly the same output as
%\begin{quote}
%\verb+\jline{[5.]{Small(a)}}+
%\end{quote}\noindent but notice that both the arguments to \verb+\nline+ are passed
%within the single argument to \verb+\jline+.
%
%To produce a justified line we need to then add the optional justification argument.
%
%\begin{verbatim}
%\jline[\ife{4,3}]{[5.]{Small(a)}}
%\end{verbatim}
%
%\noindent The \verb+\ife+ macro produces a justification using the
%$\lif$-elimination rule of the logic.  This is one of a suite of
%macros that we developed to produce justifications in a uniform
%manner.  These are described in section~\ref{sect:justifications}.
% 
%As an alternative, the |\pline| command can be used to typeset proof
%lines. It takes the formula as a mandatory argument, and line number
%and justification as optional arguments: |\pline|\oarg{line
%number}\marg{formula}\oarg{justification}.  The ariant |\fpline|
%works the same, except it produces a focus slider.
%
%\section{Logical Formulae}
%\label{sect:formulae}
%
%We set our formulae using sans serif font and we also need to be in math mode
%because the use of logical connectives.  The \verb+\formula+ command takes
%a formula to set as its argument, and handles both font and mode
%change.
%
%Many of the commands that follow require that they are used in a
%formula, or math, context.
%
%\subsection{Connectives}
%
%We introduce our own (usually) shorter names for logical connectives.  All of these
%commands require a formula context.
%
%\begin{center}
%\begin{tabular}{lc}
%\bf Usage & \bf Output\\
%\hline\\
%\verb+\land+ & \formula{\land}\\
%\verb+\lor+ & \formula{\lor}\\
%\verb+\lif+ & \formula{\lif}\\
%\verb+\liff+ & \formula{\liff}\\
%\verb+\lnot+ & \formula{\lnot}\\
%\verb+\lfalse+ & \formula{\lfalse}\\
%\verb+\lall+ & \formula{\lall}\\
%\verb+\lis+ & \formula{\lis}
%\end{tabular}
%\end{center}
%
%\subsection{Quantified formulae}
%
%So that we don't have to think about spacing when setting quantified
%formulae, we defined the following commands.
%
%\begin{verbatim}
%\newcommand{\quant}[3]{#1 #2\;#3}
%\newcommand{\exi}[2]{\quant{\lis}{#1}{#2}}
%\newcommand{\uni}[2]{\quant{\lall}{#1}{#2}}
%\end{verbatim}
%The first, \verb+\quant+ is a helper macro, but you might like to use
%\verb+\exi{x}{P(x)}+ to obtain \formula{\exi{x}{P(x)}}, and
%\verb+\uni{x}{\exi{y}{(P(x)\lif Q(x,y))}}+ for
%\formula{\uni{x}{\exi{y}{(P(x)\lif Q(x,y))}}}, for example.  \verb+\exi+ and
%\verb+\uni+ require a formula context.
%
%\section{Justifications}
%\label{sect:justifications}
%
%Here are the macros that we have developed to typeset justifications within a proof.
%These do not require a
%formula context.
%
%\begin{center}
%\begin{tabular}{ll}
%\bf Usage & \bf Output\\
%\hline\\
%|\landi{2, 3}| &\landi{2, 3}\\
%\verb+\lande{4}+&\lande{4}\\
%\verb+\lori{5}+ & \lori{5}\\
%\verb+\lore{6}{7--9}{11--13}+ &\lore{6}{7--9}{11--13}\\
%\verb+\lnoti{14--15}+ &\lnoti{14-15}\\
%\verb+\lnote{16}+ &\lnote{16}\\
%\verb+\lfalsei{17}{19}+ &\lfalsei{17}{19}\\
%\verb+\lfalsee{20}+&\lfalsee{20}\\
%\verb+\lifi{21--25}+ &\lifi{21--25}\\
%\verb+\life{27}{30}+ &\life{27}{30}\\
%\verb+\liffi{33--37}{40-50}+&\liffi{33--37}{40--50}\\
%\verb+\liffe{55}{57}+ &\liffe{55}{57}\\
%\verb+\reit{4}+&\reit{4}\\
%\verb+\eqi+ &\eqi\\
%\verb+\eqe{7}{11}+&\eqe{7}{11}\\
%\verb+\lalli{9--12}+&\lalli{9--12}\\
%\verb+\lalle{13}+&\lalle{13}\\
%\verb+\lexii{15}+ &\lexii{15}\\
%\verb+\lexie{17}{18-30}+ &\lexie{17}{18--30}\\
%\end{tabular}
%\end{center}
%
%\section{Other Macros}
%\label{sect:other-macros}
%
%\subsection{Arguments}
%
%When you want to typeset an argument, for example when setting an exercise requiring
%the proof of some conclusion from a collection of premises, you should use the
%\verb+\fitcharg+ command.  This command takes two arguments, both mandatory, the
%premise lines which are set above the Fitch bar, and the conclusion lines, which go
%below.  Here's an example.  The argument
%
%\begin{quote}
%\fitcharg{\formula{P(a)}\\ \formula{Q(a)}}{\formula{S(a)}}
%\end{quote}
%
%\noindent is typeset by the command:
%\begin{verbatim}
%\fitcharg{
%          \formula{P(a)}\\ 
%          \formula{Q(a)}
%         }
%         {\formula{S(a)}}
%\end{verbatim}
%\noindent Notice that multiple lines are separated by a use of the \LaTeX\ command
%\verb+\\+.
%
%\subsection{Fitch Contexts}
%
%The \verb+\fitchctx+ command is used for setting proof fragments which do not
%include the horizontal Fitch bar.  We use these, for example, when formally
%describing the inference rules of our logic.  The command below, for example,
%typesets our description of the rule of conditional proof.
%
%Notice here that we use the command \verb+\ellipsesline+ to produce a line of the
%``proof'' containing a vertical ellipses (using the \verb+\nline{\vdots}+ command,
%which is the obvious solution, results in the ellipses not being centered on the
%line).
%
%Another new feature of this situation is the use of the \verb+\fline+ command. 
%This command takes a single, mandatory, arguments which consists of all of the
%arguments to a \verb+\jline+ command.  It produces exactly what the \verb+\jline+
%command would, except that a focus slider appears to the left of the Fitch bar in
%question.  The \verb+\fline+ does not work in the argument lines of \verb+\fitchprf+
%or \verb+\fitcharg+.
%
%\begin{verbatim}
%\fitchctx{
%     \subproof{\nline[$n$.]{P}}
%              {\ellipsesline\\
%               \nline[$m$.]{Q}
%              }
%     \fline{[\lifi{$n-m$}]{{P\lif Q}}}
%}
%\end{verbatim}
%
%\begin{quote}
%\fitchctx{
%     \subproof{\nline[$n$.]{P}}
%              {\ellipsesline\\
%               \nline[$m$.]{Q}
%              }
%     \fline{[\lifi{$n-m$}]{{P\lif Q}}}
%}
%\end{quote}
%
%
%
% \StopEventually{}
%
% \section{Implementation}
%
%    \begin{macrocode}
\NeedsTeXFormat{LaTeX2e}
\ProvidesPackage{lplfitch}[2013/05/07 -- Fitch Proofs a la LPL]
%    \end{macrocode}
%
% \begin{macro}{\formula}
% Typesets a formula (in math mode, letters typeset in sans-serif as in LPL
%    \begin{macrocode}
\providecommand{\formula}[1]{\ensuremath{\sf{#1}}}
%    \end{macrocode}
% \end{macro}
%
% \subsection{Connectives and Quantifiers}
% We provide convenient and short commands for logcal symbols.
% These first three are appropriately defined in LaTeX2e, and 
% therefore do nothing except remind people that they are there.
%    \begin{macrocode}
\providecommand\land{\wedge}
\providecommand\lor{\vee}
\providecommand\lnot{\neg}
\providecommand\lif{\rightarrow}
\providecommand\liff{\leftrightarrow}
\providecommand\lfalse{\bot}
\providecommand\lall{\forall}
\providecommand\lis{\exists}
%    \end{macrocode}
% The following provide commands for ommands for setting quantified formulae, inserting correct space
% between variable and formula.
%    \begin{macrocode}
\providecommand{\quant}[3]{#1 #2\;#3}
\providecommand{\exi}[2]{\quant{\lis}{#1}{#2}}
\providecommand{\uni}[2]{\quant{\lall}{#1}{#2}}
%    \end{macrocode}
% \subsection{Justifications}
%
% The folowing commands are used to generate justifications, e.g., |\landi{1, 2}|
% produces ``\landi{1, 2}''.
%    \begin{macrocode}
\providecommand\intro[1]{\formula{#1\,}{\bf Intro:}}
\providecommand\elim[1]{\formula{#1\,}{\bf Elim:}}

\providecommand\landi[1]{\intro{\land} #1}
\providecommand\lande[1]{\elim{\land} #1}
\providecommand\lori[1]{\intro{\lor} #1}
\providecommand\lore[3]{\elim{\lor} #1, #2, #3}
\providecommand\lnoti[1]{\intro{\lnot} #1} 
\providecommand\lnote[1]{\elim{\lnot} #1}

\providecommand\lfalsei[2]{\intro{\lfalse} #1, #2}
\providecommand\lfalsee[1]{\elim{\lfalse} #1}

\providecommand\lifi[1]{\intro{\lif} #1}
\providecommand\life[2]{\elim{\lif} #1, #2}

\providecommand\liffi[2]{\intro{\liff} #1, #2}
\providecommand\liffe[2]{\elim{\liff} #1, #2}

\providecommand\reit[1]{{\bf Reit:} #1}

\providecommand\eqi{\intro{=}}
\providecommand\eqe[2]{\elim{=} #1, #2}

\providecommand\lalli[1]{\intro{\lall} #1}
\providecommand\lalle[1]{\elim{\lall} #1}

\providecommand\lexii[1]{\intro{\lis} #1}
\providecommand\lexie[2]{\elim{\lis} #1, #2}
%    \end{macrocode}
%
% \subsection{Dimensions}
%
% The following dimensions can be redefined with |\setlength|
%
% \begin{macro}{\fitchargone}
% Width of formula in arguments 
%    \begin{macrocode}
\newdimen\fitchargone   \fitchargone 3.5in
%    \end{macrocode}
% \end{macro}
%
% \begin{macro}{\fitchone}
% Width of formulas in proofs
%    \begin{macrocode}
\newdimen\fitchone   \fitchone 3.0in
%    \end{macrocode}
% \end{macro}
%
% \begin{macro}{\fitchtwo}
% Distance between scope line and formula
%    \begin{macrocode}
\newdimen\fitchtwo   \fitchtwo 10pt
%    \end{macrocode}
% \end{macro}
%
% \begin{macro}{fitchthree}
% distance between scope line and formula  ; RZ
%    \begin{macrocode}
\newdimen\fitchthree \fitchthree=\fitchtwo
\advance\fitchthree by 0pt
%    \end{macrocode}
% \end{macro}
%
% \subsection{Contexts, Arguments, Proofs}
%
% \begin{macro}{\fitchctx} Typesets fitch context, e.g., specification of 
% a rule, in which a focus slider is typeset to the left of the conclusion 
%    \begin{macrocode} 
\providecommand{\fitchctx}[1]{\advance \fitchone by -\fitchthree%
\advance \fitchone by .5pt
\begin{tabular}[t]{r@{}|p{\fitchone}@{}l}
 \phantom{\slider}  \\[-1.75ex]
 #1 \\[-1.75ex] & & 
 \end{tabular}
\advance \fitchone by \fitchthree%
}
%    \end{macrocode}
% \end{macro}
%
% \begin{macro}{\fitchprf} Typesets a fitch proof
%    \begin{macrocode}
\providecommand{\fitchprf}[2]{\advance \fitchone by -\fitchthree%
\advance \fitchone by .5pt
\hspace*{.35em}\begin{tabular}[t]{|p{0pt}@{}p{\fitchone}@{}l}
 \multicolumn{3}{@{}l@{}}{\ }\\[-2.35ex]
  #1 \\ 
  \ \\[-2.5ex] \cline{1-1}\\[-2ex]
  #2 \\ \multicolumn{3}{@{}l@{}} \ \\[-2.35ex]
 \end{tabular}
\advance \fitchone by \fitchthree%
}
%    \end{macrocode}
% \end{macro}
% \begin{macro}{\fitcharg} Typesets an argument (without line numbers)
%    \begin{macrocode}
\providecommand{\fitcharg}[2]{\advance \fitchargone by -\fitchthree%
\hspace*{.35em}
\begin{tabular}[t]{|p{\fitchtwo}@{}p{\fitchargone}}
 \multicolumn{2}{@{}l@{}}{\ }\\[-2.35ex]
  #1 \\ 
  \ \\[-2.5ex] \cline{1-1}\\[-2ex]
  #2 \\ \multicolumn{2}{@{}l@{}} \ \\[-2.35ex]
\end{tabular}
}
%    \end{macrocode}
% \end{macro}
%
% \begin{macro}{\slider} Typesets the focus arrow.
%    \begin{macrocode}
\providecommand{\slider}{\mbox{$\triangleright \;$}}
%    \end{macrocode}
% \end{macro}
% 
% \begin{macro}{\pline} Typesets a proof line with an optional line number (before the 
% formula argument) and an optional justification (after the formula argument)
% E.g.  |\pline[5.]{Cube(a)}[\lande{6}]|
%    \begin{macrocode}
\def\pline{\@ifnextchar[\@plinenum{\@plinenum[\@empty]}} 
\def\@plinenum[#1]#2{\@ifnextchar[{\@plinex{#1}{#2}}{\@plinex{#1}{#2}[\@empty]}} 
\def\@plinex#1#2[#3]{ & #1\formula{\; #2} & #3} 
%    \end{macrocode}
% \end{macro}
%
% \begin{macro}{\fpline} Like |pline| except producesa focus slider on the left
%    \begin{macrocode}
\def\fpline{\slider\pline} 
%    \end{macrocode}
% \end{macro}
%
% \begin{macro}{\tline} Typesets proof line containing text wiht an optional 
% line number.
% E.g. |\tline[1.]{\formula{a} is a cube.}|
%    \begin{macrocode}
\providecommand{\tline}[2][{}]{ & #1\hspace{.6em}{#2}}
%    \end{macrocode}
% \end{macro}
%
% The following macros |nline|, |jline|, and |fline|
% are the original commands provided in the original version
% of |lplproof| and are included for backwards compatibility. 
% E.g. |\nline[1.]{P}|
%    \begin{macrocode}
\providecommand{\nline}[2][{}]{ & #1\formula{\;#2}}
%    \end{macrocode}
% justified line (justification optional)
% |\jline[justification]{args for line}|
%    \begin{macrocode}
\providecommand{\jline}[2][{}]{\nline#2 & #1}
%    \end{macrocode}
% focused line
%    \begin{macrocode}
\providecommand{\fline}[1]{\slider \jline#1}
%    \end{macrocode}
%
% \begin{macro}{\ellipsesline}
% Producing vertical ellipses in a line on their own.
% This takes care of centering the ellipses in their line.
%    \begin{macrocode}
\providecommand{\ellipsesline}[0]{\nline{\;\raise.65ex\hbox{\vdots}}}
%    \end{macrocode}
% \end{macro}
% \begin{macro}{\subproof}
% This command sets a subproof within a proof.
% The arguments are the lines before, and the lines after
% the horizontal fitch bar.
%    \begin{macrocode}
\providecommand{\subproof}[2]{&\fitchprf{#1}{#2}\\}
%    \end{macrocode}
% \end{macro}
% \begin{macro}{\boxedsubproof}
% This produces a new boxed subproof.
% |\boxedsubproof[n]{c}{premise}{body}|
% Where n is the number of the premise line
%       c is the constant(s) to be boxed
%       premise is the premise of the subproof (a formula)
%       body is the lines of the subproof.
%    \begin{macrocode}
\providecommand{\boxedsubproof}[4][{}]{
     \subproof{\nline[#1]{\fbox{\formula{#2}}\;#3}}{#4}
}
%    \end{macrocode}
% \end{macro}
% \begin{macro}{\brokenform}
% Crude attempt at dealing with formula which need to be broken to fit
% in the proof.  We take two arguments, the first line of the formula
% which is left justified, and the remainder of the lines, which
% are right justified.
%    \begin{macrocode}
\providecommand{\brokenform}[2]{
   \advance\fitchone by-\fitchthree
   \begin{tabular}[c]{rp{\fitchone}}
    \multicolumn{1}{p{\fitchone}@{}}{\formula{#1}}\\
    \formula{#2}
   \end{tabular}
   \advance\fitchone by\fitchthree
}
%    \end{macrocode}
% \end{macro}
%
% \Finale
\endinput
